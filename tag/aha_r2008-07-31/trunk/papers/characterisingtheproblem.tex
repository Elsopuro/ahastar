\section{Characterising the problem}
In the literature, we seldom find any examples of planners that deal with variable-sized agents planning efficiently in multi-terrain environments. Despite the importance of this problem, it seems most works either avoid the problem entirely or only solve part of it. Perhaps this is because, superficially, this problem appears trivial: given a grid-based representation of the map, any tile which is not an obstacle is traversable for some agent. Even in complex non-homogenous environments it is not difficult to imagine a planner which takes into consideration an agent's terrain-traversal capability and limits the search space to locations the agent is able to cross. \\ \newline
The problem manifests its difficulty when we introduce variability in the size of the agent. Each location in a grid environment has a fixed size; if some agents are larger than others they could occupy several tiles at once and the performance of our planner would quickly degrade as we are forced to check for collisions with static obstacles at every time-step. Increasing the resolution of the grid seems a promising possibility. Unfortunately, because each location on a grid map can have only a single terrain type, a coarse-grain abstraction applied in this fashion would result in paths which are much longer than optimal. Another problem, particularly for smaller agents, is un-realistic movement stemming from too-early obstacle avoidance. \\ 
A more promising approach is to create a multi-level grid; one for each size of agent. This ensures quality paths for all agents but leads to an exponential blow-up in memory as we increase the number and type of agents and sizes. \\ \newline
In order to solve this problem it appears we will need a more robust strategy that will enable us to deal with large degrees of heterogeneity. We will borrow and extend ideas from several existing works and show how we can achieve high quality path planning in such circumstances.


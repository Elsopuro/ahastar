\section{Related Work}
\label{sec:relatedwork}
Symmetry is often an undesirable characteristic in a search space.
In the presence of symmetry, search algorithms can evaluate many equivalent
states and make little real progress toward the goal.
The problem of how best to deal with symmetry has received significant attention 
in areas such as planning \cite{fox99} and constraint programming \cite{gent00} 
but there are very few works that explicitly identify and deal with symmetry in 
pathfinding domains such as grid maps. 
\par
\citeauthor{pochter10}~\shortcite{pochter10} describe a search space reduction
algorithm which bears some similarity to RSR.  Their approach involves
decomposing a map into so called \emph{Swamp} areas that can be ignored because
there always exists a symmetric path that does not involve any nodes from the swamp.
Each search instance is then limited to a corridor of provably necessary swamp
and non-swamp areas. All remaining nodes in the graph are ignored. 
The focus of this approach is therefore on 
identifying regions of the search space that can be ignored.  By comparison, RSR
tries to reduce the search effort involved in exploring any given area. Thus the
two methods have complementary aims.
\par
A similar graph pruning technique, known as as the \emph{dead-end heuristic},
is due to \citeauthor{bjornsson06}~\shortcite{bjornsson06}.  Like RSR, this method
decomposes grid maps into obstacle-free zones connected by entrances and exits.
A preliminary online search in the decomposed graph is then used to identify
zones that cannot lead to the goal.  The central idea is to avoid exploring such
``dead ends'' during a subsequent search in the original grid.  This approach is
similar to the one employed by \citeauthor{pochter10}~\shortcite{pochter10} but
reported results suggest it is not as effective.


\section{Problem definition}
A \emph{gridmap} is a structure composed of square cells, of unit size, each of which represent a unique location in the environment. 
Each grid cell is an \emph{octile} connected to $n$ neighbours, where  $0 \leq n \leq 8$. 
Furthermore, each octile is associated with a particular terrain type and there are $r \geq 1$ possible terrains.
\par \indent
Each grid map is representable as a graph, $G = (V, E)$ where each grid cell corresponds to a vertex in the set of vertices $V$ and each cell/neighbour adjacency is represented by an edge in the set of all edges $E$
\par \indent
Each map tile is either blocked or traversable. 
Each traversable tile has an associated terrain type. 
Blocked tiles are called \emph{hard obstacles}, since no agent can occupy them.
\par \indent
An \emph{agent} is any entity attempting to move across a grid environment. 
Every agent is square in shape and has a size $s \geq 1$.
While stationary or moving, each agent occupies $s^2$ tiles. 
When an agent occupies several tiles simultaneoulsly, the top-left corner of the agent corresponds to its current location.
\par \indent
Every agent has a terrain traversal \emph{capability}, defined as an attribute comprising a non-empty set of terrains. 
An agent can never occupy a tile whose terrain type is not included in its capability.
\par \indent 
A \emph{soft obstacle} is a traversable tile which is not traversable by a specific agent because it lacks the correct capability or its size is larger than the associated clearance value of the tile, as defined below. 
\par \indent
A \emph{clearance value} is an obstacle-distance metric associated with a particular tile in the grid environment. 
Each clearance value measures the maximal size of an agent at a given location without intersecting any obstacle in the environment. 
A tile can have several clearance values associated with it, one for each capability. 
\par \indent
A \emph{problem instance} is defined as a pair of locations in the environment. These correspond to an initial location where an agent begins and a goal location which it actively tries to reach. A problem is valid if at least one path exists between the start and goal locations comprising only tiles traversable by the agent. 

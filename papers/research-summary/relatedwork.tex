\section{Related Work}
\label{sec:relatedwork}
Symmetry is often an undesirable characteristic in a search space.
In the presence of symmetry, search algorithms can evaluate many equivalent
states and make little real progress toward the goal.
The problem of how best to deal with symmetry has received significant attention 
in areas such as planning \cite{fox99} and constraint programming \cite{gent00} 
but there are very few works that explicitly identify and deal with symmetry in 
pathfinding domains such as grid maps. 
\par
\citeauthor{pochter10}~\shortcite{pochter10} describe a search space
reduction algorithm called \emph{Swamps} which bears some similarity to RSR.
This approach involves decomposing a map into so called ``swamp'' areas that can
be ignored because there always exists a symmetric path that does not cross any
nodes in the swamp. Each search instance is then limited to an a corridor of
inter-dependent swamp (and non-swamp areas) and all remaining nodes in the graph
are ignored. This algorithm is rather different to RSR in that the focus appears
to be on identifying regions of the search space that can be ignored.  
By comparison, RSR aims to reduce the search effort involved in exploring any
given area. Thus the two methods are in some sense complementary.
\par
RSR is also similar to the \emph{dead-end heuristic}; a recent graph pruning
technique due to \citeauthor{bjornsson06}~\shortcite{bjornsson06}.
The authors describe a method which, like RSR, decomposes grid maps into
obstacle-free zones connected by entrances and exits.  A preliminary online
search in the decomposed graph is then used to identify zones that do not appear
on any path between the start and goal node.  The basic idea is similar in some
ways to the one employed by \citeauthor{pochter10}~\shortcite{pochter10} but
reported results suggest it is not as effective.


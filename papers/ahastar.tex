\documentclass[conference]{IEEEtran}
% If the IEEEtran.cls has not been installed into the LaTeX system files,
% manually specify the path to it.  e.g.
% \documentclass[conference]{./IEEEtran} 

% Add and required packages here
\usepackage{graphicx,times}
\usepackage{amsmath, amssymb}
\usepackage{algorithm}
\usepackage[noend]{algorithmic}
\usepackage{url}

\newtheorem{theorem}{\vspace{5mm}Theorem}
\newtheorem{lemma}[theorem]{Lemma}
\newtheorem{proposition}[theorem]{Proposition}
\newtheorem{corollary}[theorem]{Corollary}
\newtheorem{definition}[theorem]{Definition}

% Correct bad hyphenation here
\hyphenation{optical networks semiconductor IEEEtran}

% To create the author's affliation portion using \thanks
\IEEEoverridecommandlockouts
				
\textwidth 178mm
\textheight 239mm
\oddsidemargin -7mm
\evensidemargin -7mm
\topmargin -6mm 
\columnsep 5mm

\begin{document}

% Paper title: keep the \ \\ \LARGE\bf in it to leave enough margin.
\title{Hierarchical path planning for multi-size agents in heterogenous environments}
\author{\textbf{Daniel Harabor, Adi Botea} \\ National ICT Australia and The Australian National University \\ $\lbrace$ daniel.harabor $|$ adi.botea $\rbrace$ at nicta.com.au}

% Make the title area
\maketitle

\input abstract
\input introduction
\input relatedwork
\input problemdefinition
\input computingclearance
\input annotatedastar
\input mapabstraction
\input dominance_relations
\input hierarchicalplanning 
\input experimentalsetup
\input results
\input conclusion
\input acknowledgements

\bibliographystyle{IEEEtran}
\bibliography{references}

% Trigger a \newpage just before a given reference number in order to
% balance the columns on the last page.  Adjust the value as needed;
% it may need to be readjusted if the document is modified later.
%\IEEEtriggeratref{8}
% The "triggered" command can be changed if desired:
%\IEEEtriggercmd{\enlargethispage{-5in}}

% The references section can either be generated by hand or by an
% automatic tool like BibTeX.  If using BibTex, use the standard IEEEtran
% bibliography style.
%\bibliographystyle{IEEEtran.bst}
%
% The argument to \bibliography is/are the name(s) of your BibTeX file(s)
% that contains string definitions and bibliography database(s).
%\bibliography{IEEEabrv,SamplePaper}
%
% If you generate the bibliography by hand, or if you copy in the
% resultant .bbl file, set the second argument of \begin to the number of
% references in the bibliography (used to reserve space for the reference
% number labels box).
\end{document}


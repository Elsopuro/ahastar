\section{Identifying Empty Rooms}
\label{empty rooms}
In this section we give a simple but effective flood-fill-based algorithm for decomposing a 
grid map into empty rectangular rooms.
Our method attempts to identify large open spaces by examining the amount of \emph{clearance}
(a distance-to-obstacle metric) around a tile and growing that area until some termination 
condition is reached.
%In Algorithm \ref{alg-clearance} we give a recursive procedure for computing clearances.
%This technique is originally described in \cite{harabor08} where clearances are used to
%facilitate pathfinding for multi-size agents navigating across grid maps.
In order to compute clearances we employ the technique described in \cite{harabor08}.
However, for reasons of brevity, we will omit a description of that algorithm.
%\input alg_clearance
\par
Once clearance values are computed for each tile on the map we apply the following
procedure to construct empty rectangular rooms:


\begin{enumerate}
\item{Iterate over all traversable nodes on the map, identifying nodes that have not yet 
been assigned to a room.}
\item{Starting with the first unassigned node, attempt a maximal extension of the room
by flood-filling over the nodes in the current row (to the right of the start node).}
\item{Stop flood-filling when one of the following conditions is met: an obstacle is found,
a node which is already assigned to another room is found or an unassigned
node is found which has a larger clearance value than the node at the origin of the room.
The length of the row of tiles prior to termination is the width of the room.}
\item{Repeat Steps 2-3 for each subsequent row attempting to grow the room
each time to its maximal width.
The number of rows successfully processed before the algorithm terminates is the height of the room.}
\end{enumerate}

% We found this algorithm to work significantly better than other more naive flood-fill
% approaches (for example, as described in \cite{bjornsson06,bolanca09}). 
% In particular, the early termination condition at Step 3 
% due to increasing clearance is critical to avoid the situation where
% several smaller rooms are extended into an area which could be occupied by one large room.

As we will see the performance of A* on our modified grid map is closely related to the total 
number of nodes we are able to prune.
Thus, identifying large rooms is critical.
Although our decomposition technique is not optimal for this purpose it is simple
to understand and implement and produces good results in practice.

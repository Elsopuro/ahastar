\begin{abstract}
% NB: 200 words max
Pathfinding systems that operate on regular grids are common in the AI literature
and often used in video games.
Typical speed-up enhancements include reducing the size of the search space using abstraction,
and building more informed heuristics.
Though effective each of these strategies has shortcomings. 
For example, pathfinding with abstraction usually involves trading away optimality
for speed.
Meanwhile, improving on the accuracy of the well known Manhattan heuristic usually
requires significant extra memory.
\par
We present a different kind of speedup technique based on the idea
of identifying and eliminating symmetric path segments in 4-connected grid maps 
(which allow straight but not diagonal movement).
Our method decomposes such maps into a set of rectangular rooms with no obstacles 
and then prunes all symmetric paths from the interior of each room.
The resulting map is often much smaller and consequently much faster to search
than the original.
We evaluate this technique on a range of different grid maps 
including a well known set from the popular video game Baldur's Gate 2.
Results show that A* can run up to 3.3 times faster on average
and expand 64\% fewer nodes.
We achieve this without using any significant extra memory or sacrificing solution optimality.
\end{abstract}

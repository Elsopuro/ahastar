\begin{abstract}
% NB: 200 words max
Pathfinding systems that operate on regular grids are common in the AI literature
and often used in video games.
Typical speed-up enhancements include reducing the size of the search space using abstraction,
and building more informed heuristics.
Though effective each of these strategies has shortcomings. 
For example, pathfinding with abstraction usually involves trading away optimality
for speed.
Meanwhile, improving on the accuracy of the well known Manhattan heuristic usually
requires significant extra memory.
\par
We present a different kind of speedup technique based on the idea
of identifying and eliminating symmetric path segments in 4-connected grid maps 
(which allow straight but not diagonal movement).
Our method decomposes such maps into a set of rectangular rooms with no obstacles 
and then prunes all nodes from the interior of each room.
We show that the resultant grid maps preserve the completeness and optimality 
characteristics of the original.
We then give an empirical analysis by running A* on a range of different grid maps 
including a well known set from the popular video game Baldur's Gate.
Our results show that on average A* runs between 1.7 to 3.3 times faster on our pruned 
grid maps and expands between 40-70\% fewer nodes.
\end{abstract}

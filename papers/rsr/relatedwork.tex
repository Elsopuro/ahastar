\section{Related Work}
\label{sec:relatedwork}
Dealing with symmetry is a regularly appearing topic in the literature.  It has
been addressed in areas as diverse as planning \cite{fox99}, constraint programming
\cite{gent00}, and combinatorial optimization \cite{fukunaga08}.  This is not
surprising given that, in the presence of symmetry, search algorithms often
evaluate many equivalent states and make little real progress toward the goal.
Despite receiving significant attention in other areas, there are very few
works that explicitly identify and deal with symmetry in pathfinding domains 
such as grid maps. 
\par
The work most closely related to ours is due to
\cite{harabor10}.  They introduce Empty Rectangular
Rooms, a symmetry breaking technique specific to 4-connected grid maps which we
will refer to as 4ERR.  We discuss the main differences between 4ERR and our
work in Sections \ref{sec:introduction} and \ref{sec:rsr}.
\par
Our work is also related to that of
\cite{pochter10}.  They introduce \emph{swamps}, an
alternative search space reduction algorithm which requires decomposing a graph
into so called ``swamp'' areas that can be ignored because there always exists a
symmetric path that does not cross any nodes in the swamp.  Each search instance
is then limited to an a corridor of inter-dependent swamp (and non-swamp areas)
and all remaining nodes in the graph are ignored.  The identification of swamps
is quite different to our empty rectangle decomposition and the focus of the
algorithm appears to be in identifying regions of the search space that can be
ignored.  By comparison, RSR focuses on reducing the search effort required to
explore any given area.
%Additionally, swamps are shown to be most effective in areas featuring a large number of obstacles 
%and less effective on maps featuring wide open areas.
%By comparison, our algorithm is most effective when large empty rectangles can be identified and less
%effective when this is not the case.
Thus the two methods are in some sense complementary.
\par
RSR bears some similarity to new heuristic methods aimed at improving the
performance of standard A* on grid maps \cite{bjornsson06}.  In that work, like
in ours, grid maps are decomposed into obstacle-free zones connected by
entrances and exits.  A preliminary online search in the decomposed graph
identifies zones that do not appear on any path between the start and goal node,
thus yielding the \emph{dead-end heuristic}.  It can be seen as a technique for
detecting areas that don't have to be searched in the instance at hand.
The basic idea is similar in some ways to the one employed by
\cite{pochter10} but reported results suggest it is
not as effective.


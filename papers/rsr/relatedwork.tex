\section{Related Work}
\label{sec:relatedwork}
Dealing with symmetry is a regularly appearing topic in the literature.  It has
been addressed in areas as diverse as planning \cite{fox99}, constraint programming
\cite{gent00}, and combinatorial optimization \cite{fukunaga08}.  This is not
surprising given that, in the presence of symmetry, search algorithms often
evaluate many equivalent states and make little real progress toward the goal.
Despite receiving significant attention in other areas, there are very few
works that explicitly identify and deal with symmetry in pathfinding domains 
such as grid maps. 
\par
The work most closely related to ours is due to \cite{harabor10}.  They
introduce Empty Rectangular Rooms, a symmetry breaking technique specific to
4-connected uniform-cost grid maps.  We refer to this work 4ERR and discuss the
main differences between it and our present work in Sections
\ref{sec:introduction} and \ref{sec:rsr}.
\par
The \emph{dead-end heuristic} \cite{bjornsson06} and \emph{Swamps-based
pathfinding} \cite{pochter10} are two closely-related graph pruning techniques
that involve the same central idea: to identify areas in the search space that
are not relevant for reaching the goal. This is a similar yet complementary goal
to RSR, which tries to reduce the search effort involved in exploring any given
area.  The dead-end heuristic, like RSR, proceeds by decomposing the map into a
series of obstacle-free rooms connected by entrances and exits.  A preliminary
online search in the decomposed graph is then used to identify so called ``dead
ends'' -- rooms that can be ignored during a subsequent search in the original
grid as exploring them is not necessary to reach the goal.  A stronger variant
of this idea is used in \cite{pochter10}.
This time the objective is to decompose the map into a set of so 
called \emph{Swamp} areas. Nodes in a Swamp can be ignored as there always
exists a set of symmetric paths that never cross the Swamp.  To achieve a
speedup, each search instance is limited to a corridor of provably relevant
Swamp and non-Swamp areas and all remaining nodes in the graph are ignored.
\par
The \emph{portal heuristic} \cite{goldenberg10} is a new memory-based technique
which also attempts to speed up optimal pathfinding on grid maps.  Like RSR, it
too proceeds to decompose the map into a series of adjacent areas.  It then
pre-computes and stores exact distances between all pairs of so called ``portal nodes''
i.e. nodes that transition from one area to another. The main idea is to
use this information to improve the accuracy of cost-to-go estimates during
search and in the process reduce the number states expanded by A* on the way to
the goal.  Perhaps the main difference between this approach and other similar
techniques \cite{sturtevant09,felner09,bjornsson09}, is its variable memory
requirements: by building larger or smaller areas it can control the number of
portal nodes that must be considered and thus trade heuristic accuracy for 
memory.
\par
In the algorithm engineering community the problem of quickly computing optimal
shortest paths has received significant attention.
State of the art methods such as TRANSIT \cite{bast06} and Contraction Hierarchies
\cite{geisberger08} are based
on a combination of Dijkstra's algorithm together with memory-intensive 
abstractions that reduce the search space by way of node elimination or the
introduction of short-cut edges.  Though these algorithms are very fast, they are
also highly optimised for road networks in which certain topological properties
hold true: for example, the existence of ``highway'' edges that appear on most
shortest paths between arbitrary pairs of nodes. Though these ideas are mostly
orthogonal to RSR, there has been very
little work applying them to searching on grid maps. One recent result
however \cite{sturtevant10} suggests they are not as effective when the underlying 
graph contains a high degree of path symmetry. Future work may involve 
cross-fertiziling ideas from these two areas to develop better approaches.

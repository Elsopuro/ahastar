\section{Introduction}
\label{sec:introduction}

Grid-based pathfinding systems are commonly found in academic literature; for
example in application areas such as robotics \cite{choset05} and video games
\cite{botea04,sturtevant05,bjornsson06}.  In such contexts it is often the case
that queries sent to the pathfinding system need to be solved as quickly as
possible.  Traditionally, this requirement is met through the application of
hierarchical decomposition techniques such as HPA*~\cite{botea04} and PRA*
~\cite{sturtevant05} that transform the search space into a much smaller
approximate representation.  Such methods are very fast, particularly when
compared to the classical A* algorithm, but have the disadvantage that solutions
found in the abstract state space are often not optimal when mapped back to the
original grid.  An alternative speedup method is to develop better
heuristics to guide the search~\cite{sturtevant09}.  Though usually fast and
more effective than the popular Manhattan or Octile heuristic (both analogous to
Euclidean distance but optimised to 4 and 8-connected grids), they have the
disadvantage of requiring significant memory overhead.
\par
In this paper we present Rectangular Symmetry Reduction (RSR): a graph pruning
algorithm for undirected uniform-cost grid maps which is fast, memory efficient,
optimality preserving and which can, in some cases, eliminate entirely the need
to search.  The central idea that we will explore involves the identification
and elimination of path symmetries from the search space. 
%For our purposes, we will (informally for the moment) say that two paths are
%symmetric if they are identical save for the order in which the individual moves
%(or steps) occur.  
To deal with path symmetries RSR makes use of an off-line
empty rectangle decomposition. Originally described in \cite{harabor10}, this
method converts an arbitrary grid map into an equivalent one where only nodes
from the perimeter of each empty rectangle need to be explored during search.
We extend \cite{harabor10} in several directions: (i) we generalise the method
from 4-connected grid maps to the 8-connected case where the increase in
branching factor makes effective symmetry elimination more challenging; (ii) we
develop a new offline pruning technique that reduces the number of nodes which
need to be explored during search; (iii) we give a novel online pruning strategy
which speeds up node expansion by selectively evaluating either all neighbours
associated with a particular node or only a small subset.  We prove that in each
case both optimality and completeness are preserved.
\par
We perform a thorough empirical analysis, comparing RSR with three similar
state-of-the-art graph pruning algorithms on a
number of synthetic and realistic benchmarks, including one well known set from
the popular roleplaying game \emph{Baldur's Gate II}.  Compared to Harabor and
Botea's method \cite{harabor10}, we both extend the applicability and improve
the speed on the subset of instances where both algorithms are applicable.
We also compare RSR to the recent Swamps-based pruning method of Pochter \emph{et
al}~\cite{pochter10} and the enhanced Portal Heuristic of Goldenberg \emph{et
al}~\cite{goldenberg10}. 
We show that RSR has complementary strengths compared to both of these methods 
and identify classes of instances RSR is clearly the better choice, 
dominating convincingly across a large number of instances.


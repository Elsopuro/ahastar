\section{Introduction}
Single-agent path planning is a well known and extensively studied problem in computer science.
It has many applications such as logistics, robotics, and more recently, computer games. 
Despite the large amount of progress that has been made in this area, to date, very little work has focused specifically on addressing planning for diverse-size agents in heterogeneous terrain environments. 
\par \indent
The problem is interesting because such diversity introduces much additional complexity when solving path planning problems.
Modern real-time strategy or role-playing games for example often feature a wide array of units of differing shapes and abilities that must contend with navigating across environments with complex topographical features -- many terrains, different elevations etc. 
Thus, a route which might be valid for an infantry-solider may not be valid for a heavily armoured tank. 
Likewise, a car and an off-road vehicle may be similar in size and shape but the paths preferred by each one could differ greatly. 
\par \indent
Unfortunately, the majority of current path planners, including recent hierarchical planners (\cite{botea04}, \cite{sturtevant05}, \cite{demyen07}, \cite{geraerts07}), only perform well under certain ideal conditions. 
They assume, for example, that all agents are equally capable of reaching most areas on a given map and any terrain which is not traversable by one agent is not traversable by any. 
Further assumptions are often made about the size of each respective agent; a path computed for one is equally valid for any other because all agents are typically of uniform size. 
Such assumptions limit the applicability of these techniques to solving a very narrow set of problems: homogeneous agents in a homogeneous environment. \\
We address the opposite case and show how efficient solutions can be calculated in situations where both the agent's size and terrain traversal capability are variable. 
Our method extends recent work emerging from the areas of robotics and computer games which has shown the effectiveness of using clearance annotations to measure obstacle distance at key locations in the environment and using this information to help agents plan better paths. 
\par \indent
We contribute in several ways: first, we introduce AHA*, a new clearance-based hierarchical path planner; second, we show how to leverage clearance in order to produce compact yet information rich search abstractions; third, we provide a detailed empirical analysis of our new technique on a wide range of problems involving multi-size agents in heterogeneous multi-terrain environments.
\par \indent
The rest of this paper is organised as such: first, we cover existing work in the area of hierarchical path planning and multi-size agent search. We then define the problem and describe our map annotation approach before showing how to adapt A* to plan for a range of agents of different sizes and capabilities. We go on to detail a new map abstraction technique that leverages map annotations and characterise the worst-case performance. In the final sections we introduce our hierarchical planner, AHA*, and provide a detailed analysis of its performance before concluding. 

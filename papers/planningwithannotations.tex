\section{Annotated A*}
\label{aha:aastar}
Having defined a technique for annotating a graph with pertinent information about important topographical features in the environment, we turn our attention to how one may leverage our newfound information richness to discover suitable paths for diverse sets of agents. \\ \newline
We use a variation on the well-known A* algorithm \cite{astar} to compute an optimal shortest path between a start and goal node.  Our approach differs from the standard implementation by requiring two additional parameters for each query: minimum clearance and a set of traversable terrains; these correspond to the size and capability attributes of the agent under consideration. \\ \newline
The fundamental insight is simple: we compare the size and capabilities of our agent with the corresponding terrain clearance annotations of each node evaluated by A*. A traversable node must have a clearance value at least equal to the size of the agent. By making sure our planner only considers locations which meet this criteria we can limit our agents to navigating inside an obstacle-free corridor whose width is defined by the minimum terrain clearance in the set of nodes along the optimal path between the start and goal. This is a similar approach used by the  CMM planner \cite{geraerts07} discussed earlier but we deal with a more complex multi-terrain case. \\ \newline
We thus introduce Annotated A* (AA* for short); a low-level path planner for multi-size agents in heterogenous terrain environments which we outline in algorithm \ref{alg_aastar}

%\input content/alg5_annotatedastar

It is important to note that AA* is a more specialised version of A*; we constrained the original problem using size and terrain clearance as parameters guiding search. We did not however change the way A* behaves. AA* uses the same evaluation function as its simpler counterpart and therefore retains the same completeness and optimality characteristics and, as we will later see in our empirical analysis, the number of nodes expanded by our planner remains, in the worst case, exponential in the length of the optimal solution -- again, just like A*.

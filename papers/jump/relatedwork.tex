\section{Related Work}
\label{sec:relatedwork}
The academic literature is rich in works that deal with the problem of symmetry
in search.  Approaches for identifying and eliminating symmetry have been
proposed in areas as as diverse as planning \cite{fox99}, constraint programming
\cite{gent00}, and combinatorial optimization \cite{fukunaga08}. Until recently
however, very few works explicitly identify and deal with symmetry in pathfinding
domains such as grid maps.
%This is not surprising given that, in the
%presence of symmetry, search algorithms often evaluate many equivalent states
%and make little real progress toward the goal.  Despite receiving significant
%attention in other areas, there are very few works that explicitly identify and
%deal with symmetry in pathfinding domains such as grid maps.
\par
One recent study which attempts to redress this oversight is due to
\citeauthor{harabor10}~\shortcite{harabor10}.  They introduce Empty Rectangular
Rooms, a symmetry breaking technique specific to 4-connected grid maps.  Using
this method a grid map is  decomposed into a set of obstacle-free rectangles.
Nodes from the interior of each rectangle are then pruned and replaced by a
series of macro edges that facilitate optimal travel from any node on the
perimeter of a rectangle to any other.  Being specific to 4-connected maps this
approach is less general than our jump point technique. It also requires offline
pre-processing whereas our method is online.
\par
Related to our work is the graph pruning technique of
\citeauthor{pochter10}~\shortcite{pochter10}.  They introduce \emph{swamps}, a
search space reduction algorithm which involves decomposing a graph into so
called ``swamp'' areas that can be ignored because there always exists a
symmetric path that does not cross any nodes in the swamp.  Each search instance
is then limited to a corridor of inter-dependent swamp (and non-swamp areas).
The focus of swamp pruning is in identifying regions of
the search space which can be ignored for the search at hand.  This is in
contrast to our work which focuses on reducing the effort required to explore
any given area in the search space.  Another difference is that swamp
identification is usually performed offline while we identify jump points
online.
\par
Searching with jump points bears some similarity to the graph pruning work of
\citeauthor{bjornsson06}~\shortcite{bjornsson06}.  These authors introduce the
\emph{Dead-end Heuristic}; an optimality preserving
 method for improving the performance of A* on grid maps.
As with Swamps and Empty Rooms, an offline pre-processing step is required where
the map is decomposed into a set of zones.  The main idea is to identify areas
that can be ignored for the search instance at hand by running a preliminary
search in the decomposed graph.  The approach has similar goals to the one
employed by \citeauthor{pochter10}~\shortcite{pochter10} but reported results
suggest it is not as effective.
\par
Hierarchical pathfinding methods are pervasive in cases where optimality is
not required.  The main idea is to improve performance by reducing the 
search space (usually a grid map) into a much smaller approximation.  Algorithms
of this type are  usually fast and memory-efficient but also suboptimal.  The HPA*
algorithm \cite{botea04} is one such example.  It uses a map decomposition
approach, dividing a grid map into a series of fixed-size clusters connected by
entrances.

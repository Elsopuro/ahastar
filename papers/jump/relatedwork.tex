\section{Related Work}
\label{sec:relatedwork}
The academic literature is rich in works that deal with the problem of symmetry
in search.  Approaches for identifying and eliminating symmetry have been
proposed in areas as as diverse as planning \cite{fox99}, constraint programming
\cite{gent00}, and combinatorial optimization \cite{fukunaga08}. Until recently
however, very few works explicitly identify and deal with symmetry in pathfinding
domains such as grid maps.

A recent algorithm which attempts to redress this oversight is Empty Rectangular
Rooms~\cite{harabor10}: an offline symmetry breaking technique specific to
4-connected uniform-cost grid maps.  The main idea is to decompose the search
space into a series of obstacle-free rectangles and prune all nodes from the interior
of each rectangle. These are replaced by a series of macro edges that
facilitate optimal travel from any node on the perimeter of a rectangle to any
other.  Being specific to 4-connected maps this approach is less general than
our jump point technique. It also requires offline pre-processing whereas our
method is online.

Also related to our work are two similar search space reduction techniques: the
\emph{dead-end heuristic} \cite{bjornsson06} and \emph{Swamps} \cite{pochter10}.
Both methods use an offline pre-processing step to decompose a grid into a
series of adjacent areas.  Later, a preliminary online search in the decomposed
graph identifies areas that do not have to be explored in order to optimally
reach the goal.  This objective is similar yet orthogonal to our work where
the aim is to reduce the effort required to explore any given area in the search
space.

A different method for pruning the search space is to identify \emph{dead} and
\emph{redundant} cells~\cite{sturtevant10}.  Developed in the context of
learning-based heuristic search, this method is used to speed up pathfinding in
real-time settings.  Unlike jump points however, a benefit is only obtained
after running multiple iterations of an iterative deepening algorithm (each
identifying new pruning opportunities).  Further, the identification of
redundant cells requires additional memory overheads which jump points do not
have.

A recent suggestion for speeding up optimal A* search is to \emph{fast expand}
nodes~\cite{sun09}.  This approach avoids unnecessary open list operations by
identifying and immediately expanding successor nodes which are just as good or
better than the node currently at the top of the open list.  Jump points are a
similar yet fundamentally different idea: they allow us to identify large sets
of nodes that would be ordinarily expanded but which can be skipped entirely.
 
In cases where optimality is not required, hierarchical pathfinding methods
are pervasive.  The main idea is to improve performance by decomposing the
search space (usually a grid map) into a much smaller approximation.  Algorithms
of this type, such as HPA*~\cite{botea04}, are  usually fast and
memory-efficient but also suboptimal.

\section{Experimental Setup}
We evaluate the performance of RSR on four benchmarks taken from the freely
available pathfinding library Hierarchical Open Graph
(HOG)\footnote{\url{http://www.googlecode.com/p/hog2}}: 
\begin{itemize}

\item{\textbf{Adaptive Depth}
is a set of 12 maps of size 100$\times$100 in which approximately $\frac{1}{3}$
of each map is divided into rectangular rooms of varying size and a large
open area interspersed with large randomly placed obstacles.
} 

\item{\textbf{Baldur's Gate} is a set map set taken from BioWare's popular
roleplaying games \emph{Baldur's Gate II: Shadows of Amn}.
Appearing regularly as a standard benchmark in the literature
\cite{bjornsson06,harabor10,pochter10}, 
it consists of 120 maps of various sizes which have all been scaled to
$512\times512$. 
}

\item{\textbf{Dragn Age} is another realistic benchmark; this time taken from
BioWare's recent roleplaying game \emph{Dragon Age: Origins}.
This is a newer benchmark than Baldur's Gate but has appeaed previously in the
literature; e.g. \cite{sturtevant07}. It consists of 156 maps which range in size
}


\item{\textbf{Rooms} is a set of 300 maps of size
256$\times$256 which are divided into symmetric rows of small rectangular areas
($7\times7$), connected by randomly placed entrances. This benchmark has
previously appeared in \cite{pochter10}.
}
\end{itemize}

As there are no problem instances provided with Adaptive Depth and Rooms
we randomly generated 100 valid problems per map.
Both Baldur's Gate and Dragon Age comes a range of pre-generated problem 
instances; 93160 total in the case of the former, and 159507 for the latter. 
\par
Our test machine is a 2.93GHz Intel Core 2 Duo processor with 4GB RAM running OSX 
10.6.4.  Our implementation of A* is based on one provided in HOG, which we 
adapted to facilitate online neighbour pruning and jump point identification. 

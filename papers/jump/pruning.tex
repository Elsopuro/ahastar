\section{Pruning Rules}
Grid maps are highy regular problem domains. 
In particular, each node in the grid has a fixed number of neighbours (usually
8 but nodes on the perimeter of the map or next to an obstacle have less) and each
neighbour is always located in the same position relative to the current node.
We exploit this regularity to develop pruning rules that speed up individual 
node expansion operations by proving that not all neighbours of the current node 
need to be evaluated. 
Consider for example Figure \ref{fig:pruningrules}(a), where the search is
expanding node $x$ by following a straight edge \footnote{We distinguish between
straight edges and diagonal edges, depending on the type of transition being
made.} from $p(x)$.
In this case we would normally have to evaluate all 8 neighbours of $x$.
However, if we consider the direction of travel from $p(x)$ to $x$, we can 
immediatly prune the following moves:
\renewcommand{\descriptionlabel}[1]%
{{\labelsep}\textsf{#1}}
\begin{enumerate}
\item[4    ] This move returns to $p(x)$ with cost $(c + 1)$; 
yet we already expanded this node earlier with cost $c$. 
\item[1 (7)] Each of these moves has cost $(c + 1) + \sqrt2$ yet both neighbours
can be reached with cost $c + 1$ fom $p(x)$.
\item[2 (8)] Each of these moves has cost $(c + 1) + 1$ yet both neighbours can
be reached with cost $c + \sqrt2$ from $p(x)$.
\item[3 (9)] Each of these moves has cost $(c + \sqrt2) + \sqrt2$; however this is
symmetric to an alternative path via $p(x)$ which does not mention $x$.
Note that this is only true if neighbours 2 (8) are not blocked. 
When this is not the case, we must consider
\end{enumerate}

Notice that in this case we eliminate 7 of the possible 8 neighbours; leaving
only move 6 to be evaluated. However, this assumes that move 2 (8) is not
blocked. If this is the case we must also 

A similar process can be applied when following a diagonal edge from $p(x)$ to 
$x$; this case is illustrated in Figure \ref{fig:pruningrules}(b).
Using the same kind of reasoning as before we argue that only moves 2, 3 and 6
need to be evaluated and the rest can be discarded:
\begin{enumerate}
\item[7    ] This move returns to $p(x)$ with cost $(c + \sqrt2)$ when we already
expanded this node with cost $c$.
\item[4 (8)] Each of these moves has a cost $(c + \sqrt2) + 1$ yet both
neighbours can be reached with cost $c + 1$ from $p(x)$.
\item[1 (9)] Each of these moves has a cost $(c + \sqrt2) + \sqrt2$ yet both
can be reached via an alternative path from $p(x)$ with cost $c + 2$;
\end{enumerate}

Up to this point we have assumed that when $x$ is expanded all its neighbours
are present and traversable. This is not true in all cases; for example, $x$ may
have fewer than 8 neighbours if it is located on the perimeter of a map or
adjacent to an obstacle.
In the case of a straight transition, 

A special case of this pruning rule appears when one or more of the neighbours
of $x$ are not traversable. This scenario is highlighted in Figure \ref{fig:pruningrules}(b). 

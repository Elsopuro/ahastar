\section{Notation and Terminology}
\label{sec:notation}
We work with undirected uniform-cost grid maps.
Each node has $\leq 8$ neighbours and is either traversable or not. 
A straight move, from a traversable node to one of its neighbours, 
has a cost of 1; diagonal moves cost approx. $\sqrt 2$.
Moves involving non-traversable (obstacle) nodes are disallowed.

A path $\pi =~<n_{0}, n_{1}, \ldots , n_{k}>$ is a
cycle-free ordered walk starting at node $n_{0}$ and ending at 
$n_{k}$.
We will sometimes use the setminus operator in the context of a path:
for example $\pi \setminus x$. This means that the subtracted node $x$
does not appear on (i.e. is not mentioned by) the path. 
We will also use the function $len$ to refer the length of a path 
and the function $d$ to refer to the distance between two nodes 
on the grid: e.g. $len(\pi)$ or $d(n_{0}, n_{k})$ respectively.
As we work with only straight-line distances, applying $d$ requires no search.




\section{Notation and Terminology}
\label{sec:notation}
We work with undirected uniform-cost grid maps.
Each node has $\leq 8$ neighbours and is either traversable or not. 
Straight moves, i.e., horizontal or vertical moves, 
from a traversable node to one of its neighbours, 
have a cost of 1; diagonal moves cost $\approx\sqrt 2$.
Moves involving non-traversable (obstacle) nodes are disallowed.
The notation $\vec{d}$ refers to one of the eight moves.  
We write $y = x + k\vec{d}$ when node~$y$ can be reached 
by taking $k$ unit moves from node~$x$ in direction $\vec{d}$.
If move~$\vec{d}$ is diagonal, 
then the two straight moves $45\deg$ to $\vec{d}$ are denoted by $\vec{d_1}$ and $\vec{d_2}$.  

A path $\pi =~\begin{pth}n_{0}, n_{1}, \ldots , n_{k}\end{pth}$ is a
cycle-free ordered walk starting at node $n_{0}$ and ending at 
$n_{k}$.
We will sometimes use the setminus operator in the context of a path:
for example $\pi \setminus x$. This means that the subtracted node $x$
does not appear on (i.e. is not mentioned by) the path. 
We will also use the function $len$ to refer the length (or cost) of a path 
and the function $d$ to refer to the distance between two nodes 
on the grid: e.g. $len(\pi)$ or $d(n_{0}, n_{k})$ respectively.
%As we work with only straight-line distances, applying $d$ requires no search.




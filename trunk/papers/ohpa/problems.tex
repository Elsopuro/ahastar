\section{Problems}
This section contains unresolved issues that we need to figure out.
It will need to be deleted at some point.
\subsection{Search effort could be more than A*}
On an empty map, hierarchical paths could involve every cluster. 
Using an A* with unlucky tie-breaking we expand every node in every cluster and do this process for each hierarchical path during refinement. 
This could expand more nodes than A*.
\par
One suggestion is to keep a shared closed list during the refinement process. 
However, Daniel believes this will not be preserve optimality (ie. a node closed during the hierarchical refinement of one path could be on the optimal path for another yet-to-be-refined hierarchial path)

\subsection{Unclear how to represent an entrance as a single node}
In HPA, an entrance has transition points and each transition point is represented in the graph by 2 nodes.
This works nicely because each side of the entrance results in a node in the hierarchical graph and each node in the hierarchical graph is associated with only one cluster.
In this work, we propose to use a single node to represent the entrance. 
This presents problems when inserting the start and goal into the graph -- how do you know which cluster the entrance belongs to? 
Another problem is that an entrance also has an associated traversal cost (the cost of transitioning from one cluster to another) which isn't being factored in.
\par
One easy fix is to use one node to represent each half of a HPA* entrance (the bits unique to one cluster).

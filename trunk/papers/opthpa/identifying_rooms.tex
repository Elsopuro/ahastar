\section{Identifying Empty Rooms}
\label{empty rooms}
In this section we give a simple but effective flood-fill-based algorithm for decomposing a 
grid map into empty rectangular rooms.
We will try to build large rooms before small ones and prefer rooms which
contain as many interior nodes as possible:

\begin{enumerate}

\item{\label{rooms_alg:step1} For each traversable tile $t$, build a maximal size empty
rectangle which has $t$ as its upper left corner. Each such rectangle should contain only
traversable tiles which have not already been assigned to a room.}

\item{\label{rooms_alg:step2} Using a Max-Heap, sort the list of traversable tiles using the number of
interior nodes in the rectangle of each $t$ as its priority.}

\item{\label{roomsalg:step3} Take from the heap the tile $t$ with highest priority
which has not already been assigned to a room. }

\item{\label{rommsalg:step4} Verify the priority of $t$ by building another maximal size
empty rectangle (as per Step \ref{roomsalg:step1}) which has $t$ as its upper
left corner and contains no obstacles or tiles already been assigned to another room.}

\item{\label{roomsalg:step5} If the number of interior nodes in the new rectangle is equal to the 
priority of $t$ we say that the rectangle forms a room and add it to our decomposition. 
Otherwise, we update the priority of $t$ with the number of interior 
nodes contained in the new rectangle. }

\item{\label{step6} Repeat Steps \ref{roomsalg:step3} to \ref{roomsalg:step5} until the heap is
empty and all nodes have been assigned to a room in the decomposition.}

\end{enumerate}

The construction of empty rooms is similar to the computation of \emph{clearance
values} in \citeauthor{harabor08}~\shortcite{harabor08}. 
In this case however we work with maximal size rectangles instead of squares. 
\par
As we will see the performance of A* on our modified grid map is closely related to the total 
number of nodes we are able to prune.
Thus, identifying large rooms is critical.
Although our decomposition technique is not optimal for this purpose it is simple
to understand and implement and produces good results in practice.

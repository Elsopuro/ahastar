\section{Related Work}
A very effective method for solving path planning problems is to reformulate the original problem
into an equivalent one in a much smaller abstract search space. 
Known as hierarchical path planning this approach has been well studied over the last decade;
algorithms in this category are usually fast, memory-efficient and suboptimal.
OHA* follows this general idea while seeking to retain optimality.
It is perhaps most similar to the popular HPA* algorithm \cite{botea04} 
which divides a grid map into a series of fixed-size clusters connected 
by entrances.
Finding a shortest path then involves inserting the start and goal into the 
abstract graph and searching for a (usually suboptimal) path between them.
\par
A recent method which shares quite a few similarities with OHA* is MSA* \cite{bolanca09}.
MSA* operates on 8-connected grid maps and, like OHA*, 
attempts to speed up search by exploiting path equivalence in empty rooms.
However, where OHA* prunes nodes entirely from the interior of empty rooms 
MSA* uses macro successors to limit the number of interior nodes that must 
be visited during a room traversal.
A speedup over conventional A* is reported but the algorithm is also
shown to expand a large number of interior nodes which hampers its performance.
It is also worth noting that MSA* uses a different decomposition 
technique to OHA* in order to identify empty rooms.
\par
OHA* also bears some similarity to a recent result outlining new heuristic methods for improving the 
performance of standard A* on games maps \cite{bjornsson06}.
In that work, much like in ours, grid maps are decomposed into obstacle-free zones connected by entrances 
and exits. 
A preliminary search in the decomposed graph identifies zones that do not appear 
on any path between the start and goal node, thus yielding the \emph{dead-end heuristic} 
(a similar yet orthogonal idea to our main result).
A highly informed (and memory intensive) \emph{gateway heuristic} is also developed which 
requires computing and storing exact costs between all pairs of entrances and exits.
Both heuristics are shown to be admissible and yield optimal length paths. 
Average running time improvements over standard A* are reported at between 10-40\% using a range of
realistic games maps.
\par
Fringe Search is yet another algorithm which attempts to preserve optimality
while improving over standard A*.
This work is quite different from others we have discussed in that it is not 
hierarchical and it does not rely on the development of memory-intensive heuristics.
Rather, Fringe Search is a general purpose iterative deepening technique with memory requirements
comparable to A*. 
It is provably optimal if maximum search depth is sufficiently large and, crucially, has been shown to 
run between 25-40\% faster than A*.
As with the dead-end heuristic, Fringe Search is orthogonal to OHA* and the two
can easily be combined.

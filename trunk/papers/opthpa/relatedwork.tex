\section{Related Work}
%Symmetry arises in many problems in artificial intelligence; from planning and scheduling
%domains to general constraint satisfaction and even pathfinding.
%In almost all cases symmetry is undesirable as it increases the size of the search space
%and wastes time that could be spent finding better solutions.
%Our symmetry elimination technique bears some similarity to the work of Walsh \cite{walsh07}
%which focuses on breaking variable assignment symmetries in constraint satisfaction problems.
%By comparison, we study the much more specific problem of eliminating path symmetries that 
%appear between pairs of nodes on a grid map. 
%\par
One recent result on the problem of symmetric path elimination in grid maps is due to
Pochter et al \cite{pochter09}.
Their main contribution is a technique for the identification of swamps, which are defined as
areas on a grid map that contain only nodes appearing on symmetric paths.
Though irregular in both shape and size, swamps can be thought of as a generalisation of our
empty room decomposition.
One important difference however is that swamps are shown to be most effective on maps featuring a
large number of obstacles and less effective on maps featuring wide open areas.
By comparison, our algorithm is most effective when large empty rooms can be identified and less
effective when this is not the case.
Thus the two methods are in some sense complimentary.
Another important difference is that swamps are identified online and thus introduce additional 
overhead to a running search.
By comparison, our pruning method is offline and has no such overhead.
Finally, keeping track of swamps requires an additional memory overhead.
By comparison, our method has no overhead and in many cases actually reduces the amount of
memory required to store the map.
\par
Our work also bears some similarity to new heuristic methods aimed at improving the 
performance of standard A* on grid maps \cite{bjornsson06}.
In that work, much like in ours, grid maps are decomposed into obstacle-free zones connected by entrances 
and exits. 
A preliminary search in the decomposed graph identifies zones that do not appear 
on any path between the start and goal node, thus yielding the \emph{dead-end heuristic} 
(a similar yet orthogonal idea to our main result).
A highly informed (and memory intensive) \emph{gateway heuristic} is also developed which 
requires computing and storing exact costs between all pairs of entrances and exits.
Both heuristics are shown to be admissible and yield optimal length paths. 
Average running time improvements over standard A* are reported at between 10-40\% using a range of
realistic games maps.
\par
MSA* \cite{bolanca09} is a new optimality preserving search algorithm which attempts to speed up search 
on 8-connected grid maps by exploiting path equivalance in empty rooms. 
Rather than pruning nodes from the interior of an empty room however MSA* attempts to speed up 
search by generating macro edges on the fly.
An improvement over conventional A* is reported but the algorithm is also
shown to expand a large number of nodes from the interior of empty rooms, which hampers its performance.
It is also worth noting that MSA* uses a different empty room decomposition method
from the one described in our work.
\par
Fringe Search \cite{bjornsson06} is a general purpose iterative deepening technique which also
aims to improve on the performance of A*.
This work is quite different from others we have discussed in that it does not
rely on any specific decomposition technique nor on the development of any new heuristics
to guide the search.
It is provably optimal if maximum search depth is sufficiently large and, crucially, has been shown to 
run between 25-40\% faster than A*.
As with the dead-end heuristic, Fringe Search is orthogonal to our work and the two
can easily be combined.
\par
Another very effective method for solving path planning problems is to reformulate the original problem
into an equivalent one in a much smaller abstract search space. 
Algorithms in this category are usually fast, memory-efficient and suboptimal.
Our empty room decomposition is inspired, in part, by work in this area and is perhaps most similar to 
the popular HPA* algorithm \cite{botea04} 
which divides a grid map into a series of fixed-size clusters connected 
by entrances.
As with Fringe Search, HPA* is orthogonal to our work and the two could be easily combined.

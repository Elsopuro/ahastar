\section{Identifying Empty Rooms}
\label{empty rooms}
In this section we give a simple but effective flood-fill-based algorithm for decomposing a 
grid map into empty rectangular rooms.
We will try to build large rooms before small ones and prefer rooms which
contain as many interior nodes as possible:

\begin{enumerate}
\item{For each traversable tile $t$, build an empty rectangle which has $t$ as its upper-left corner and which
contains as many interior nodes as possible.}
\item{\label{step2} Using a Max-Heap, sort the list of traversable tiles using the number of
interior nodes in the rectangle of each $t$ as its priority.}
\item{\label{step3} Take from the heap the tile $t$ with highest priority and try to build an empty
room. }
\item{\label{step4} If $t$ is already assigned to a room, discard it. 
Otherwise, try to build an empty room with $t$ as its upper-left corner.}
\item{\label{step5} If the number of nodes in the empty room is at least equal to the priority 
of $t$ we add the room to our decomposition. }
\item{\label{step6} If the number of nodes in the empty room at $t$ is not at least equal to
the priority of $t$ we update the priority of $t$ and put it back the heap.}
\item{\label{step7} Repeat Steps \ref{step3} to \ref{step6} until the heap is
empty and all nodes have been assigned to a room in the decomposition.}
\end{enumerate}

The construction of empty rooms is similar to the computation of \emph{clearance
values} in \citeauthor{harabor08}~\shortcite{harabor08}, however we work with rectangles
instead of squares. 
\par
As we will see the performance of A* on our modified grid map is closely related to the total 
number of nodes we are able to prune.
Thus, identifying large rooms is critical.
Although our decomposition technique is not optimal for this purpose it is simple
to understand and implement and produces good results in practice.

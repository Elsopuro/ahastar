\section{Rectangular Symmetry Reduction}
\label{sec:currentwork}
I have previously looked at the problem of symmetry breaking in 4 and 8-connected
grid maps.
Both appear regularly in the literature \cite{yap02} and are often found in the 
pathfinding systems of modern video games; e.g BioWare's  \emph{Dragon Age:
Origins} (PC) and Square Enix's \emph{Children of Mana} (Gameboy Advance).
\par 
In \cite{harabor10} my co-author and I propose the following offline strategy 
for identifying and eliminating symmetric paths in 4-connected grid maps:
\begin{enumerate}
\item{Decompose the grid map into a set of empty rooms, where each empty room is 
rectangular in shape and free of any obstacles. 
The size of the rooms can vary across a map, depending
on the placement of the obstacles.}
\item{Prune all nodes from the interior but not the perimeter of each empty
room.}
\item{Add a series of \emph{macro edges} that connect each node on the perimeter of an empty room
with a node on the directly opposite side of the room 
\footnote{Alternatively, macro edges could be generated on-the-fly during search. 
This obviates the need to store them explicitly.}.
The cost of each edge is equal to the Manhattan distance between its two endpoints.
}
\item{To handle cases where the start or goal location is a node previously pruned, we
use an insertion procedure that re-adds single nodes back into the graph for the
duration of the search.}
\end{enumerate}
We prove the optimality of this approach and evaluate its performance by running a
wide range of experiments to measure the relative speedup experienced by A* 
when searching on pruned vs. un-pruned grids.
Results show that in many
cases over 50\% of all nodes on a given map can be pruned, resulting in
improvements to average search times by a factor of up to 3.5.
\par
In \cite{harabor11a} we propose Rectangular Symmetry Reduction (RSR); an 
algorithm which extends the empty rectangle decomposition idea in several directions: 
(i) we generalise the method from 4-connected grid maps to the
8-connected case where the branching factor makes effective symmetry elimination
more challenging; (ii) we develop a new offline pruning technique that reduces
the number of nodes which need to be explored during search; (iii) we give a
novel online pruning strategy which speeds up node expansion by selectively
evaluating either all neighbours associated with a particular node or only a
small subset.  We prove that in each case both optimality and completeness are
preserved.
\par
We perform a thorough empirical analysis, comparing RSR with two similar
state-of-the-art graph pruning algorithms ~\cite{pochter10,harabor10}
on a number of synthetic and realistic benchmarks.
Compared to \cite{harabor10}, we both extend the applicability and improve the speed
on the subset of instances where both methods are applicable.
Furthermore, we show that RSR and the swamp-based method of 
\citeauthor{pochter10}~\shortcite{pochter10}
have complementary strengths and identify classes of instances where
either RSR or swamps is more suitable.
We conclude that swamps are better suited for maps with
small open areas and their effectiveness reduces on maps with larger open areas.
In contrast, larger open areas allow RSR to build larger empty rectangles,
leading to a corresponding improvement in performance.
In particular, our results show a wide range of instances where
RSR is clearly the better choice, dominating convincingly the benchmark algorithms.

\section{Jump Point Search}
In \cite{harabor11b} my co-author and I develop Jump Point Search (JPS)
an online macro operator that
deals with symmetry by selectively expanding only certain nodes on a grid map
which we call \emph{jump points}.  Moving from one jump point to the next
involves travelling in a fixed direction while repeatedly applying a set of
simple neighbour pruning rules until either a dead-end or a jump point is
reached.  Because we do not expand any intermediate nodes between jump points
our strategy can have a dramatic positive effect on search performance.
Furthermore, computed solutions are guaranteed to be optimal.  
\par
Results show that JPS is many times faster than Swamps
\cite{pochter10}, a recent state-of-the-art optimality preserving pruning
technique.  We also show that JPS is competitive with, and in
many instances faster than, HPA*~\cite{botea04}; a popular though sub-optimal pathfinding
technique which is often employed in performance sensitive applications such as
video games.
\par
JPS is unique in the pathfinding literature in that it has very few
disadvantages: it is simple, yet highly effective; it preserves optimality, yet
requires no extra memory;  it is extremely fast, yet requires no preprocessing.
Further, our method is completely orthogonal to and easily combined with 
competing speedup techniques from the literature.
We are unaware of any other algorithm which has all these features.

\section{Introduction}
\label{sec:introduction}

Grid-based pathfinding systems are commonly found in academic literature; for
example in application areas such as robotics \cite{choset05} and video games
\cite{botea04,sturtevant05,bjornsson06}.  
In such contexts it is often the case that queries sent to the pathfinding
system need to be solved as quickly as possible.  Traditionally, this
requirement is met through the application of hierarchical decompostion
techniques such as HPA*~\cite{botea04} and PRA* ~\cite{sturtevant05} which
transform the search space into a much smaller approximate representation.
Such methods
can solve large problems very quickly, particularly when compared to the
classical A* algorithm, but only by sacrificing optimality.  An
alternative speedup method is to develop better heuristics to guide the
search~\cite{sturtevant09}.  Though usually fast and
more effective than the popular Manhattan or Octile heuristic (both
analogous to Euclidean distance but optimised to 4 and 8-connected grids), they
have the disadvantage of requiring significant memory overhead.

%,bjornsson06}.
%\par
%Recent work \cite{bjornsson06,pochter10,harabor10}
%has introduced techniques that aim at eliminating drawbacks
%such as a large memory overhead or solution suboptimality.
%Algorithms of this type are shown to produce 
%results that are not only optimal but also memory efficient, particularly when compared 
%with memory-based heuristics.
%Additionally, they can significantly improve the performance of well known graph
%search algorithms such as A*.
%Our contributions fit into this category.
%
%Very recently a third class of methods, based on the idea of search space
%reduction, has emerged as a promising alternative for speeding up pathfinding
%search.
%Algorithms of this type \cite{bjornsson06,pochter10,harabor10} are shown to produce 
%results that are not only optimal but also memory efficient, particularly when compared 
%with memory-based heuristics.
%Additionally, they can significantly improve the performance of well known graph
%search algorithms such as A*.
\par
In this paper we present Rectangular Symmetry Reduction (RSR): a
memory efficient, optimality preserving graph pruning algorithm which reduces
the size of the search space by identifying and eliminating path symmetries from
undirected grid maps.
RSR makes use of an off-line empty rectangle decomposition, originally described in
\cite{harabor10}, to convert an arbitrary grid map into an equivalent grid map where only nodes from the 
perimeter of each empty rectangle need to be explored during search.
%Though effective, the method is limited to 4-connected grid maps where only straight moves, and not diagonal, are
%allowed.
We extend that work in several directions: (i) we generalise the method from
4-connected grid maps to the 8-connected case where the branching factor makes
effective symmetry elimination more challenging;
(ii) we develop a new offline pruning technique that reduces the number of nodes which
need to be explored during search;
(iii) we give a novel online pruning strategy which speeds up node expansion by selectively 
evaluating either all neighbours associated with a particular node or only a small subset.
We prove that in each case both optimality and completeness are preserved.
\par
We perform a thorough empirical analysis, comparing RSR with two similar
state-of-the-art graph pruning algorithms ~\cite{pochter10,harabor10}
on a number of synthetic and realistic benchmarks, including one well known set 
from the popular roleplaying game \emph{Baldur's Gate II}.
%Our analysis allows us to identify distinct advantages over both benchmark algorithms.
Compared to Harabor and Botea's method \cite{harabor10}, 
we both extend the applicability and improve the speed
on the subset of instances where both methods are applicable.
Furthermore, we show that RSR and the swamp-based method of 
~\cite{pochter10}
have complementary strengths and identify classes of instances where
either RSR or swamps is more suitable.
We conclude that swamps are better suited for maps with
small open areas and their effectiveness reduces on maps with larger open areas.
In contrast, larger open areas allow RSR to build larger empty rectangles,
leading to a corresponding improvement in performance.
In particular, our results show a wide range of instances where
RSR is clearly the better choice, dominating convincingly the benchmark algorithms.


\section{Introduction}
Pathfinding systems which operate on regular grids are commonly found in
academic literature; for example in application areas such as robotics 
\cite{choset05} and video games \cite{botea04,sturtevant05,bjornsson06}.
%planning \cite{thayer10,DBLP:conf/aips/HernandezMSK09,DBLP:conf/aips/WangB08,DBLP:conf/aips/BulitkoBLSS07}.
\par
In the context of single-agent pathfinding, and real-time video games in particular, 
it is often the case that queries sent to the pathfinding system  need to be solved as quickly as possible.
Traditionally, this requirement is either met through the application of hierarchical
decompostion techniques or the development of improved heuristics to better guide the search.
In the case of hierarchical decomposition, algorithms such as HPA*~\cite{botea04} and PRA*
~\cite{sturtevant05} transform the search space into a much
smaller approximate representation. They can solve large problems very quickly,
particular when compared to the classical A* algorithm \cite{hart68}, but only by sacrificing optimality.
Meanwhile, in case of the improved heuristics, it has frequently been shown
that obtaining better informed results than the popular
Manhattan or Octile heuristic usually incurs significant memory overhead \cite{sturtevant09,goldberg05,Cazenave:06}. %,bjornsson06}.
%\par
%Recent work \cite{bjornsson06,pochter10,harabor10}
%has introduced techniques that aim at eliminating drawbacks
%such as a large memory overhead or solution suboptimality.
%Algorithms of this type are shown to produce 
%results that are not only optimal but also memory efficient, particularly when compared 
%with memory-based heuristics.
%Additionally, they can significantly improve the performance of well known graph
%search algorithms such as A*.
%Our contributions fit into this category.
%
%Very recently a third class of methods, based on the idea of search space
%reduction, has emerged as a promising alternative for speeding up pathfinding
%search.
%Algorithms of this type \cite{bjornsson06,pochter10,harabor10} are shown to produce 
%results that are not only optimal but also memory efficient, particularly when compared 
%with memory-based heuristics.
%Additionally, they can significantly improve the performance of well known graph
%search algorithms such as A*.
\par
In this paper we present Room-based Symmetry Reduction (RSR): a
memory efficient, optimality preserving graph pruning algorithm which reduces
the size of the search space by identifying and eliminating path symmetries from
undirected grid maps.
RSR makes use of an off-line Empty Rectangular Rooms (ERR) decomposition, originally described in
\cite{harabor10}, to convert an arbitrary grid map into an equivalent grid map where only tiles from the 
perimeter of each empty room need to be explored during search.
%Though effective, the method is limited to 4-connected grid maps where only straight moves, and not diagonal, are
%allowed.
We extend ERR in several directions: (i) we generalise the method from 4-connected grid maps to 
the much more common 8-connected case and show how the higher branching factor associated 
with this domain makes effective symmetry elimination more challenging;
(ii) we develop a new offline pruning technique that reduces the number of nodes which
need to be explored during search;
(iii) we give a novel online pruning strategy which speeds up node expansion by selectively 
evaluating either all neighbours associated with a particular tile or only a small subset.
We show that in each case both optimality and completeness are preserved.
\par
We perform a thorough empirical analysis, comparing RSR with two similar
state-of-the-art graph pruning algorithms ~\cite{pochter10,harabor10}
on a number of synthetic and realistic benchmarks, including one well known set 
from the popular roleplaying game \emph{Baldur's Gate II}.
%Our analysis allows us to identify distinct advantages over both benchmark algorithms.
Compared to Harabor and Botea's method \shortcite{harabor10}, 
we both extend the applicability and improve the speed
on the subset of instances where both methods are applicable.
Furthermore, we show that RSR and the swamp-based method of 
\citeauthor{pochter10}~\shortcite{pochter10}
have complementary strengths and identify classes of instances where
either RSR or swamps is more suitable.
We conclude that swamps are better suited for maps with
small open areas and their effectiveness reduces on maps with larger open areas.
In contrast, larger open areas allow RSR to build larger empty rooms and in such
cases there is a corresponding improvement in its search performance.
In particular, our results show a wide range of instances where
RSR is clearly the better choice, dominating convincingly the benchmark algorithms.


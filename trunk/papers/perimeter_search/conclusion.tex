\section{Conclusion}
We introduce RSR, a new search space reduction algorithm
which makes use of an Empty Rectangular Rooms decomposition
to identify and reduce path symmetries in 4 and 8 connected
grid maps.
We undertake an analysis of RSR on a range of benchmarks from the
literature and show our new algorithm is up to 36 times faster than
traditional A* running on an umodified grid map.
Further, we show that RSR is up to five times faster than two recent
state-of-the-art search space reduction techniques.
\par
We demonstrate that RSR is not only more generally applicable
than the algorithm of \citeauthor{harabor10}~\shortcite{harabor10} 
(on which it is based) but also significantly faster on the set of 
instances where both methods can be applied (i.e., 4-connected maps).
We also compare RSR with the swamps-based reduction
of \citeauthor{pochter10}~\shortcite{pochter10}.
We show that the two algorithms have complementary strengths and that there are many
instances where RSR is the better choice.
In our experiments swamps appear to be more useful in areas with small open areas
while RSR becomes more effective as larger open areas are available on a map.
\par
Our speedup technique is quite different from traditional approaches 
that include hierarchical abstraction and memory-based heuristics.
While also effective, both of these methods have their shortcomings.
For example, hierarchical algorithm such as HPA*~\cite{botea04} and 
PRA*~\cite{sturtevant07} are fast and memory-effective but produce 
sub-optimal paths.
Memory-based admissible heuristics preserve the optimality of 
solutions but carry a significant memory overhead~
\cite{sturtevant09,goldberg05,Cazenave:06,bjornsson06}.
By comparison, RSR is not only fast and optimal but also has very little
memory overhead.

%Recent work~\cite{pochter10,harabor10} has introduced
%techniques that are fast, produce optimal solutions and
%require little additional memory.
%In this paper we have introduced RSR,
%an algorithm that converts an initial map
%into a smaller search graph without sacrificing the optimality of solutions.
%Similarly to \citeauthor{harabor10}'s approach~\shortcite{harabor10},
%our algorithm uses a decomposition of a map into disjoint rectangular
%rooms with the property that no room contains obstacle tiles.
%All interior tiles and some of the perimeter tiles 
%(except for start and target locations)
%are pruned from search.


%OPTIONAL: 
%Have here a brief summary of speed-up numbers:
%Overall, the best speed-ups we have observed for RSR are blah.
%On the same data, swamps achieve a maximal speed-up of blah.
%Harabor and Botea's method has a speed-up of blah.

\par
Future work includes reducing the branching factor in RSR further through the 
development of better map decompositions and stronger online node pruning
strategies.
%We have observed cases where the number of expanded nodes is reasonably small
%but a relatively large branching factor increases the number of visited nodes significantly.
Another interesting topic is combining swamps and RSR, which tend to complement each other.
We are also interested in applying our ideas to more general types of graphs that exhibit
(local) symmetry, such as the search graphs of planning instances.

%Perimeter Search is a symmetric path elimination technique which extends the the
%work of \citeauthor{harabor10}~\shortcite{harabor10} from 4-connected grid maps
%to the much more common 8-connected case.
%We speed up optimal pathfinding by applying an offline decomposition algorithm to
%divide a grid map into a series of empty rectangular rooms.
%We then show that it is possible to traverse optimally from one side of a room
%to another without exploring any tiles from the interior of any room.
%Our experiments show that in the presence of large rooms or wide open areas we can 
%compute optimal paths very quickly: up to 16 times faster, on average for 4-connected maps and up to
%6.5 times faster, on average, for 8-connected maps.
%On less favourable map topographies we achieve more modest improvements.
%Our method is simple to understand, very effective and can be combined with
%existing speedup techniques such as memory heuristics or hierarchical methods;
%for example as described in \cite{botea04,bjornsson05,bjornsson06}. 
%\par
%One direction for future work is to investigate alternative decomposition
%algorithms which produce bigger rooms and improve the performance of the current
%method.


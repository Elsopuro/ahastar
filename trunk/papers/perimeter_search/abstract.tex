\begin{abstract}
Pathfinding is a widely studied problem in the AI literature with broad 
applicability in areas such as planning, robotics and video games. 
In this paper we explore a symmetry-based search space reduction technique
 which speeds up optimal pathfinding on uniform cost grid maps.
A pre-processing method decomposes each map into a set of empty rectangular 
rooms and removes from each such room all interior tiles and possibly some 
from along the perimeter. We then add a series of macro edges between selected 
pairs of remaining tiles to facilitate provably optimal traversal through each 
room and to any other room. In addition, we introduce on-the-fly pruning 
techniques to further speed up search.
We evaluate our technique on a range of different grid maps, including one 
well known set from the popular video game Baldur's Gate II. We show that a 
standard search algorithm such as A*, running on our modified graphs, can be 
up to an order of magnitude faster than otherwise. We achieve this result 
without requiring any significant extra memory and retain the same optimality 
guarantees as searching on an original, unmodified, search graph.
\end{abstract}

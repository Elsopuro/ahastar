\section{Related Work}
%Symmetry arises in many problems in artificial intelligence; from planning and scheduling
%domains to general constraint satisfaction and even pathfinding.
%In almost all cases symmetry is undesirable as it increases the size of the search space
%and wastes time that could be spent finding better solutions.
%Our symmetry elimination technique bears some similarity to the work of Walsh \cite{walsh07}
%which focuses on breaking variable assignment symmetries in constraint satisfaction problems.
%By comparison, we study the much more specific problem of eliminating path symmetries that 
%appear between pairs of nodes on a grid map. 
%\par
One recent result on the problem of symmetry elimination on grid maps is due to
\citeauthor{pochter10}~\shortcite{pochter10}.
They introduce \emph{swamps}, which are areas that don't have to be searched
because crossing them would not improve the length of paths.
The identification of swamps is quite different to our empty room decomposition.
Additionally, swamps are shown to be most effective in areas featuring a large number of obstacles 
and less effective on maps featuring wide open areas.
By comparison, our algorithm is most effective when large empty rooms can be identified and less
effective when this is not the case.
Thus the two methods are in some sense complementary.
%Another important difference is that swamps are identified online and thus introduce additional 
%overhead to a running search.
%By comparison, our pruning method is offline and has no such overhead.
%Also, keeping track of swamps requires an additional memory overhead.
%By comparison, our method can often reduce the amount of
%memory required to store the map.
\par
Our work also bears some similarity to new heuristic methods aimed at improving the 
performance of standard A* on grid maps \cite{bjornsson06}.
In that work, like in ours, grid maps are decomposed into obstacle-free zones connected by entrances 
and exits.
A preliminary online search in the decomposed graph identifies zones that do not appear 
on any path between the start and goal node, thus yielding the \emph{dead-end heuristic}.
It can be seen as a technique for detecting areas that don't have to be searched
in the instance at hand and, as before, is complementary to our work.
%A highly informed (and memory intensive) \emph{gateway heuristic} is also developed which 
%requires computing and storing exact costs between all pairs of entrances and exits.
%Both heuristics are shown to be admissible and yield optimal length paths. 
%Average running time improvements over standard A* are reported at between 10-40\% using a range of
%realistic games maps.
\par
MSA* \cite{bolanca09} is a new optimality preserving search algorithm which attempts to speed up search 
on grid maps by exploiting path equivalence in empty rectangular rooms. 
Rather than pruning nodes from the interior of an empty room however MSA* attempts to speed up 
search by generating macro edges on the fly.
An improvement over conventional A* is reported but the algorithm is also
shown to expand a large number of nodes from the interior of empty rooms, which hampers its performance.
It is also worth noting that MSA* uses a different empty room decomposition method
from the one described in our work.
\par
Fringe Search \cite{bjornsson05} is a general purpose graph search algorithm which also
aims to improve on the performance of A*.
This work is quite different from others we have discussed in that it does not
rely on any specific decomposition technique nor on the development of any new heuristics
to guide the search.
Rather, it uses a novel iterative deepening technique that resumes each subsequent iteration from where the last
iteration left off. 
It is provably optimal if maximum search depth is sufficiently large and 
it has been shown to run between 25-40\% faster than A*.
As our work is predominately an offline graph-pruning technique, it could be combined with any search algorithm, including
Fringe Search.
\par
Another effective method for solving path planning problems is to reformulate the original problem
into an equivalent one in a much smaller abstract search space.
Algorithms in this category are usually fast, memory-efficient and suboptimal.
The HPA* algorithm \cite{botea04} uses a map decomposition approach,
dividing a grid map into a series of fixed-size clusters connected 
by entrances.
As with Fringe Search, HPA* could be combined with our work on symmetry breaking.
For example, first apply our pruning strategy and then apply HPA* to the resulting grid map.

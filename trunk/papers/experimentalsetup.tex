\section{Experimental Setup}
We evaluated the performance of AA* and AHA* on a set of 120 octile-based maps, ranging in size from 50x50 to 320x320, which we borrowed from a popular roleplaying game.
In their default configuration the maps only featured one type of traversable terrain interspersed with hard obstacles. 
We therefore created two derivative sets (making for a total of 360 maps) where each traversable tile on every map had either 5\% or 10\% probability of being converted into a second type of traversable terrain. 
This allowed us to evaluate the algorithms in environments featuring both soft and hard obstacles.
\par \indent
For each map we generated 100 experiments by randomly creating valid problems between an arbitrarily chosen pairs of locations and some random capability.
We used two agent sizes to solve each experiment: small (occupying one tile) and large (occupying four tiles). 
Thus, a total of 72000 problem instances (360x200) were generated for each algorithm. 
All experiments were conducted on a 1.33GHz G4 processor with 768MB RAM running OSX 10.4.6.
To implement both planners we used the University of Alberta's freely available pathfinding library, HOG (www.cs.ualberta.ca/~nathanst/hog.html). 

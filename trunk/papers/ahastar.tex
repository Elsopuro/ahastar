%\documentclass[a4paper]{article}

\documentclass[letterpaper]{article}
\usepackage{aaai}
\usepackage{times}
\usepackage{helvet}
\usepackage{courier}
\usepackage{ifpdf}

\ifpdf
\pdfinfo{
/Title (Efficient path planning for multi-capable agents of heterogenous size)
/Subject (International Conference on Automated Planning and Scheduling)
/Author (Daniel, Harabor; Adi, Botea)}
\fi

\usepackage{graphicx}

\title{Efficient path planning for heterogenous agents and terrain}
\author{Daniel Harabor \and Adi Botea \\ u4272656 at anu.edu.au \\ adi.botea at nicta.com.au}

\begin{document}

\maketitle

\section{Abstract}
In this paper we present new techniques for the automated construction of compact state space representations of complex grid maps without information loss. Our approach involves the use of graph annotations to record the amount of maximal traversable space at each location on a map. We couple this with a cluster-based hierarchical abstraction technique to build highly compact yet complete representations of the original problem. We further outline the design of a new planner, Annotated Hierarchical A* (AHA*), and demonstate how a single abstract graph can be used to plan for many different agents, including different sizes and terrain traversal capabilities. 
We undertake an empirical analysis of the performance of our new planner and contrast it with a low-level variant, Annotated A* (AA*) . The experimental results show that AHA* is able to generate paths within 10% of optimal yet produces an exponential reduction in the comparartive search effort and peak memory requirements over AA*. Meanwhile, our reusable abstraction technique is able to generate equivalent representations of large problems with just 10% the space requirements of the original.

\section{Introduction}
Agents can take many shapes and sizes; from large vehicles to small, soldiers, robots and creations of fantasy; there is no limit to their diversity. In addition to their physical characteristics, agents can also be equipped with a wide array of capabilities that allow them to traverse similarly heterogenous environments. 
Modern real-time strategy or role-playing games for example often feature a wide array of units of differing shapes and abilities that must contend with navigating across environments with complex topographical features - many terrains, different elevations etc. Thus, a route which might be valid for an infantry-solider may not be valid for a heavily armoured tank. Likewise, a car and an off-road vehicle may be similar in size and shape but the paths preferred by each one could differ greatly. 
Such diversity introduces much additional complexity when solving route-finding problems; we must be able to find not just the shortest path but the most suitable path for our agent. 
Typical challenges we must consider in such situations include creating state-space representations that are rich enough to capture all the details of the environment in which our agents operate and managing the increases in complexity that is associated with the introduction of so many additional parameters. Solving the latter challenge is particularly important in application areas such as robotics and games where the amount of computational resources that can be dedicated to path planning are small.
To date, very little work has focused specifically on addressing planning for diverse-size agents in heterogenous terrain environments. The majority of current path planners, including recent hierarchical planners, all perform very well under certain ideal conditions. They assume, for example, that all agents are equally capable of reaching most areas on a given map and any terrain which is not traversable by one agent is not traversable by any. 
Further assumptions are often made about the size of each respective agent; a path computed for one is equally valid for any other because all agents are typically of uniform size. Since most planners use these assumptions as pillars for further work the applicability any given technique becomes limited to solving a very narrow set of problems: homogenous agents in a homogenous environment. We address the opposite case and show how efficient solutions can be calculated in situations where both the agent's size and terrain traversal capability are variable.

\section{Related work}
A very effective method for the efficient compuation of path planning solutions is to make the original problem more tractable by creating and searching within a smaller yet equivalent abstract space. Such hierarchical planning, involving the use of homomorphic abstraction, can be traced to \cite{holte96} and involves a four-step approach:
\begin{enumerate}
\item{Build an abstract state-space representation.}
\item{Determine the location of the initial and goal state in the abstract space.}
\item{Find the optimal path betwen then parent states of the initial and goal state. }
\item{If necessary, refine each step in the abstract plan by solving a smaller problem in the initial state space.}
\end{enumerate}
Our basic approach is identical to \cite{holte96} however we focus our efforts specifically on solving path planning problems rather than general search. \\ \newline 
Two recent hierarchial path planners that bear some resemblance to our work are describedin \cite{botea04} and \cite{sturtevant05}. The first of these, HPA*, performs grid-based decomposition of the search-space by dividing the environment into clusters and entrances which represent inter-cluster transitions. Planning involves inserting the low-level start and goal nodes into the abstract graph and finding the shortest path between them. 
PRA* on the other hand builds a multi-level search-space by abstracting cliques of nodes; the result is to narrow the search space in the original problem to a "window" of nodes along the optimal shortest-path.\newline
Both HPA* and PRA* are focused on solving planning problems for homogenous agents in homogenous-terrain environmnets and hence are not complete when either of these variables changes. We use clearance-value calculations to work out the size of the corridor connecting the start and goal nodes, thus guiding our search to only consider traversable locations. Our basic problem-solving approach however is quite similar to HPA*, which can be seen as a simplified version of AHA*. \\ \newline
One technique which bears some similarity to our work is the use of force potentials, frequently employed to help autonomous robots find a collision-free path through an environment. The basic intution is that a robot is attracted to the far-away goal and repulsed away from obstacles as it nears them. A well known method for calculating potentials is the Brushfire algorithm \cite{latombe91}, which proceeds by annotating each tile in a grid-map with the distance to the nearest obstacle. This embedded information allows the robot to calculate repulsive potentials and makes it possible to plan using gradient descent. \newline
Brushfire is similar to AHA* in that the annotations it produces allow an agent to know something about its proximity to a nearby obstacle. AHA* on the other hand explicitly calculates the maximal size of traversable space at each location. Furthermore, unlike Brushfire, AHA* does not suffer from incompleteness which can occur when repulsive forces cancel each other out and lead to the robot descending the gradient to a local minima.\\ \newline
The Corridor Map Method \cite{geraerts07} is a recently introduced path planner which also borrows heavily from the field of robotics. This technique involves building a \emph{probabilistic roadmap} to represent map connectivity. The roadmap (or backbone path) is comprised of nodes which are annotated with clearance information that indicates the radius of a maximally sized bounding sphere before an obstacle is encountered. A node on the backbone exists at every point where the topography of the map changes which allows the planner to answer queries for multi-size agents. Potential forces are used during path execution to refine the abstract plan and avoid dynamic obstacles. \newline
Similarities exist between AHA*'s approach to computing clearance and the CMM  method; both metrics are concerned with the amount of traversable space at a given location but our approach is adapted to grid environments and we compute a clearance value for each agent capability, making our method much more information rich. More significantly, our abstract graph annotation is not limited to specifying clearance at a fixed point; we annotate edges to indicate the terrain type and clearance of a corridor connecting two points not necessarily next to each other. \\ \newline
Representing an environment using navigation-meshes is increasingly popular in theliterature. Two recent planners in this category that bears some similarity to our workis Triangulation A* and Triangulation Reduction A* \cite{demyen07}. TA* makes use of a technique known as Delaunay triangulations to cover the environment with triangular polygons whose degrees are maximised. This results in an undirected graphconnected by constrained and unconstrained edges; the former being traversable and the latter not. TRA* is an extension of this approach that abstracts the triangle mesh into a structure resembling a roadmap. 
Like our approach, both TA* and TRA* are able to answer path queries for multi-size agents. In the case of TRA* the required clearance value to traverse a triangle is annotated into the edges connecting graph nodes; again, similar to our method. The abstraction approaches used by TA and TRA* are very distinctly different from our work; where we identify homogenous areas of terrain in order to build clusters their approach aims to maximise triangle size. We also have additional requirements to identify all possible transitions between two adjacent clusters; both TA* and TRA* assume a homogenous terrain.

\bibliographystyle{aaai}
\bibliography{references}

\end{document}

\section{Conclusion}
In this paper we develop a new online node pruning
strategy for speeding up pathfinding on grid maps.
Our algorithm identifies and selectively expands only certain
nodes from a grid map which we call \emph{jump points}.
Moving between jump points involves only travelling in a
fixed direction, either straight or diagonal.
We prove that intermediate nodes on a path between two
jump points never need to be expanded and ``jumping'' over them
does not affect the optimality of search.
\par
We undertake an thorough experimental evaluation of our method on
a range of synthetic and real-world bechmarks, including two taken
from the popular video games \emph{Baldur's Gate II: Shadows of Amn}
and \emph{Dragon Age: Origins}.
We show that in each case jump point pruning is many times faster 
than Swamps \cite{pochter10}, a recent state-of-the-art optimality preserving
pruning technique.
We also show that jump point pruning is competitive with, and in many instances
faster than, HPA*; a popular though sub-optimal pathfinding technique which is 
often employed in performance sensitive applications such as video games.
\par
One interesting direction for further work is to combine jump points with other
speedup techniques such as Swamps and HPA* to achieve better results.
Another direction is to generalise the jump points method to weighted grid maps.

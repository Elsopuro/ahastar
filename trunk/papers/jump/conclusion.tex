\section{Conclusion}
We introduce a new online node pruning strategy for speeding
up pathfinding on undirected uniform-cost grid maps.  Our algorithm identifies
and selectively expands only certain nodes from a grid map which we call
\emph{jump points}.  Moving between jump points involves only travelling in a
fixed direction, either straight or diagonal.  We prove that intermediate nodes
on a path between two jump points never need to be expanded and ``jumping'' over
them does not affect the optimality of search.
\par
Our method is unique in the pathfinding literature in that it has very few
disadvantages: it is simple, yet highly effective; it preserves optimality, yet
requires no extra memory;  it is extremely fast, yet requires no preprocessing.
Further, it is largely orthogonal to and easily combined with 
competing speedup techniques from the literature.
We are unaware of any other algorithm which has all these features.
\par
The new algorithm is highly competitive with related works from the literature.
When compared to Swamps~\cite{pochter10}, 
a recent state-of-the-art optimality preserving pruning
technique, we find that jump points are up to an order of magnitude faster.
We also show that jump point pruning is competitive with, and in
many instances clearly faster than, HPA*; a popular though sub-optimal pathfinding
technique often employed in performance sensitive applications such as
video games.
\par
One interesting direction for further work is to extend jump points to other
types of grids: for example using hexagons or texes~\cite{yap02}.
Like square grids, these decompositions are highly regular which suggests that
a series of equivalent neighbur pruning rules could be developed.
As the branching factor on these grid types is lower than square grids, we posit
that jump points could be even more effective than we observed in the current paper.
Another interesting direction is to combine jump points with other
speedup techniques such as Swamps and HPA* to achieve better results.

\section{Introduction}
%The most popular pathfinding methods, such as A*, Bellman-Ford and Dijkstra's
%algorithm, and Dijkstra's algorithm, have all been developed in the context of
%general purpose problem graphs.  Finding a shortest path usually requires only
%that the start and goal locations belong to the same connected component, that
%the graph contains no negative weight edges and, in the case of A*, that an
%admissible heuristic is available.  These same algorithms however are usually
%deployed on problem domains which are more constrained: undirected uniform-cost
%graphs where each node corresponds to a location on a 2D Euclidean plane.
%Perhaps the most common example of such a domain is the ubiquitous grid map.

Widely employed in areas such as robotics \cite{lee09}, artificial intelligence
\cite{wang09} and video games \cite{davis00,sturtevant07}, the ubiquitous
undirected uniform-cost grid map is a highly popular method for representing
pathfinding environments.  Regular in nature, this domain typically features a
high degree of path symmetry \cite{harabor10,pochter10}.  Symmetry in this case
manifests itself as paths (or path segments) which share the same start and end
point, have the same length and are otherwise identical save for the order in
which moves occur.  Unless handled properly, symmetry can force search
algorithms to evaluate many equivalent states and prevents real progress toward
the goal.

In this paper we deal with such path symmetries by developing a macro operator
that selectively expands only certain nodes from the grid, which we call
\emph{jump points}. Moving from one jump point to the next involves travelling
in a fixed direction while repeatedly applying a set of simple neighbour pruning
rules until either a dead-end or a jump point is reached.  Because we do not
expand any intermediate nodes between jump points our strategy can have a
dramatic positive effect on search performance.  Furthermore, computed solutions
are guaranteed to be optimal.  Jump point pruning is fast, requires no
preprocessing and introduces no memory overheads. It is also largely orthogonal to many
existing speedup techniques are applicable to grid maps.

We make the following contributions: (i) a detailed description of the jump
points algorithm; (ii) a theoretical result which shows that searching with jump
points preserves optimality;  (iii) an empirical analysis comparing our method
with two state-of-the-art search space reduction algorithms.  We run experiments
on a range of synthetic and real-world benchmarks from the literature.
We show that jump points improve the search time performance of standard A* by
an order of magnitude and more.  We also report significant improvement over
Swamps~\cite{pochter10}, a recent optimality preserving pruning technique, and
performance that is competitive with, and in many cases dominates,
HPA*~\cite{botea04}; a well known sub-optimal pathfinding algorithm.

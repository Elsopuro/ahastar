\section{Introduction}
% DH: I feel this section needs better motivation; it should discuss why
% the 2D Euclidean constraint is important to our work: i.e. we use the
% triangle inequality together with the regular properties of grids 
% (which always have a fixed number of neighbours in the same location relative
% to any given node) to identfy pruning opportunities. 

The most popular and successful pathfinding algorithms, such as A*, Bellman-Ford
and Dijkstra's Algorithm have all been developed in the context of general
purpose problem graphs.  Finding a shortest path in such domains usually
requires only that the start and goal locations belong to the same connected
component, that the graph contains no negative weight edges and, in the case of
A*, that an admissible heuristic is available.  It is often the case however
that these same algorithms are usually deployed on problem domains which are
more constrained: undirected positively weighted graphs where each node
corresponds to a location on a 2D Euclidean plane.
%The latter restriction is commonly exploited by researchers in order to develop
%more efficient search algorithms; for example in the context of the TSP. 
%By making use of the triangle inequality, researchers are able to identify 
%pruning opportunities which ... <something>
Perhaps the most common example of such a domain is the ubiquitous grid map.

Frequently the object of academic study \cite{yap02,botea04,bjornsson06} and
widely employed in application areas such as robotics \cite{choset05} and video
games \cite{sturtevant07}, grid maps are highly regular domains which
typically feature a high degree of path symmetry \cite{harabor10,pochter10}.
Symmetry in this case manifests itself as paths (or path segments) which share
the same start and end point, have the same length and are otherwise identical
save for the order in which moves occur.
%; Figure \ref{fig:symmetry} shows a typical example.  
Symmetry is usually undesirable as it forces search algorithms
to waste time discovering identical solutions and prevents real progress toward
the goal.

In this paper we deal with path symmetries by developing a macro step move
operator that selectively expands only certain nodes on a grid map which
we call \emph{jump points}. Moving from one jump point to the next involves 
travelling in a fixed direction while repeatedly applying a set of simple
neighbour pruning rules until either a dead-end or a jump point is reached.
Because we do not expand any intermediate nodes between jump points
our strategy can have a dramatic positive effect on search performance.
Furthermore, computed solutions are guaranteed to be optimal.
Unlike other recent methods for speeding up optimal search, such as the Dead-end
Heuristic
\cite{bjornsson05}, Empty Rectangular Rooms \cite{harabor10} and Swamp-based
pathfinding \cite{pochter10}, our algorithm is applied online and requires no
pre-processing.  Additionally, unlike many sub-optimal techniques that make use 
of abstraction to improve performance (e.g. HPA* \cite{botea04} and PRA*
\cite{sturtevant05}), we require
no additional memory.

We make the following contributions: (i) a detailed description of the 
jump points algorithm; (ii) a theoretical result which shows that searching
with jump points preserves optimality;  (iii) an empirical analysis 
comparing our method with a number of recent search space reduction
algorithms.
We run experiments on both synthetic and real-world 
benchamarks, including two taken from the popular video games \emph{Dragon Age:
Origins} and \emph{Baldur's Gate}.  
Our results indicate that searching with jump points achieves more than an 
order of magnitude performance improvement when compared to standard A* search. 
We also report significant performance gains over the current state of the art
including, in many cases, performance competitive with a well known sub-optimal 
pathfinding algorithm.
Perhaps most encouraging however is that our technique is largely orthogonal to 
all existing speedup techniques which are applicable to grid maps.

%In contrast to other techniques for representing
%low-dimensional Euclidean planes, such as roadmap methods \cite{geraerts05} or
%navigation mesh approaches \cite{demyen07}, grid maps are highly regular, less
%sparse and typically require more memory to store.  However, grid maps have
%several advantages that other map representations do not. Perhaps most obvious
%is their simplicity; it is trivial to overlay a grid map over a
%Euclidean plane for the purposes of finding shortest paths between points.
%There also exist a number of well known and highly popular admissible
%heuristics, such as Manhattan distance, which are specific to grid maps.
%Finally, grid maps are guaranteed to have a fixed branching factor which
%provides constant-time guarantees for the amount of time required to expand any
%particular node during search. 
% 

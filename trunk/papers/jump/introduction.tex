\section{Introduction}
% DH: I feel this section needs better motivation; it should discuss why
% the 2D Euclidean constraint is important to our work: i.e. we use the
% triangle inequality together with the regular properties of grids 
% (which always have a fixed number of neighbours in the same location relative
% to any given node) to identfy pruning opportunities. 

The most popular and successful pathfinding algorithms, such as Bellman-Ford,
Dijkstra's Algorithm and A*, have all been developed in the context of general
purpose problem graphs.  Finding a shortest path in such domains usually
requires only that the start and goal locations belong to the same connected
component, that the graph contains no negative weight edges and, in the case of
A*, that an admissible heuristic is available.  It is often the case however
that these same algorithms are usually deployed on problem domains which are
much less relaxed: undirected positively weighted graphs where each node
corresponds to a location on a 2D Euclidean plane.
%The latter restriction is commonly exploited by researchers in order to develop
%more efficient search algorithms; for example in the context of the TSP. 
%By making use of the triangle inequality, researchers are able to identify 
%pruning opportunities which ... <something>
Perhaps the most common example of such a domain is the ubiquitous grid map.

Frequently the object of academic study \cite{yap02,botea04,bjornsson06} and
widely employed in application areas such as robotics \cite{choset05} and video
games \cite{sturtevant07,jurney07}, grid maps are highly regular domains which
typically feature a high degree of path symmetry \cite{harabor10,pochter10}.
Symmetry in this case manifests itself as paths (or path segments) which share
the same start and end point, have the same length and are otherwise identical
save for the order in which moves occur; Figure \ref{fig:symmetry} shows a
typical example.  Symmetry is usually undesirable as it forces search algorithms
to waste time discovering identical solutions and prevents real progress toward
the goal.

In this paper we study a novel search strategy for dealing with path symmetries
on grid maps. Our approach involves the development of a series of simple
pruning rules for speeding up individual node expansion operations.  In
particular, we identify and discard those neighbours that cannot appear on
 the optimal path and neighbours that can be reached with the same cost via
some alternative path involving the parent of the current node.  Unlike other
recent methods for speeding up optimal search, such as the Dead-end Heuristic
\cite{bjornsson05}, Empty Rectangular Rooms \cite{harabor10} and Swamp-based
pathfinding \cite{pochter10}, our method is applied online and requires no
pre-processing.  We also require no additional memory overhead, which is
commonly the case with sub-optimal techniques that make use of abstraction to
improve performance; for example HPA* \cite{botea04} and PRA*
\cite{sturtevant05}.

We give a detailed description of our new method and a theoretical analysis that
proves its optimality preserving characteristics.  We then evaluate its
effectiveness on a number of synthetic and real-world benchamrks taken from the
literature, including two from the popular video games \emph{Dragon Age:
Origins} and \emph{Baldur's Gate}.  Our results show that the method is highly
effective; we achieve an more than a ten times improvement over standard A* and
and report significant performance gains over the current state of the art.
Perhaps most encouraging is that our technique is largely orthogonal to all
existing speedup techniques which are applicable to grid maps.

%In contrast to other techniques for representing
%low-dimensional Euclidean planes, such as roadmap methods \cite{geraerts05} or
%navigation mesh approaches \cite{demyen07}, grid maps are highly regular, less
%sparse and typically require more memory to store.  However, grid maps have
%several advantages that other map representations do not. Perhaps most obvious
%is their simplicity; it is trivial to overlay a grid map over a
%Euclidean plane for the purposes of finding shortest paths between points.
%There also exist a number of well known and highly popular admissible
%heuristics, such as Manhattan distance, which are specific to grid maps.
%Finally, grid maps are guaranteed to have a fixed branching factor which
%provides constant-time guarantees for the amount of time required to expand any
%particular node during search. 
% 
